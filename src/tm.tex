%!TEX root = ts.tex
\setcounter{section}{9}
\rSec0[lex]{Lexical conventions}


\rSec1[lex.name]{Identifiers}

\pnum
In 5.10 {[}lex.name{]}, add \uline{\texttt{atomic}} to table 4
{[}tab:lex.name.special{]}.

\rSec0[basic]{Basics}
\rSec1[basic.exec]{Program execution}
\rSec2[intro.execution]{Sequential execution}

\pnum
Change in 6.9.1 {[}intro.execution{]} paragraph 5:

\begin{quote}
A \emph{full-expression} is

\begin{itemize}
\tightlist
\item
  ...
\item
  an invocation of a destructor generated at the end of the lifetime of
  an object other than a temporary object (6.7.7) whose lifetime has not
  been extended, \sout{or}
\item
  \uline{the start and the end of an atomic block (8.8 {[}stmt.tx{]}),
  or}
\item
  an expression that is not a subexpression of another expression and
  that is not otherwise part of a full-expression.
\end{itemize}
\end{quote}

\rSec2[intro.multithread]{Multi-threaded executions and data races}
\rSec3[intro.races]{Data races} 
\pnum
Change in 6.9.2.1 {[}intro.races{]} paragraph 6:

\begin{quote}
\uline{Atomic blocks as well as} \sout{Certain} \uline{certain} library
calls \uline{may} \emph{synchronize with} other \uline{atomic blocks
and} library calls performed by another thread.
\end{quote}

\pnum
Add a new paragraph after 6.9.2.1 {[}intro.races{]} paragraph 20:

\begin{quote}
An atomic block that is not dynamically nested within another atomic
block is termed a transaction. {[}Note: Due to syntactic constraints,
blocks cannot overlap unless one is nested within the other.{]} There is
a global total order of execution for all transactions. If, in that
total order, a transaction T1 is ordered before a transaction T2, then

\begin{itemize}
\tightlist
\item
  no evaluation in T2 happens before any evaluation in T1 and
\item
  if T1 and T2 perform conflicting expression evaluations, then the end
  of T1 synchronizes with the start of T2.
\end{itemize}

{[} Note: If the evaluations in T1 and T2 do not conflict, they might be
executed concurrently. -\/- end note {]}
\end{quote}

\begin{quote}
Two actions are \emph{potentially concurrent} if ...
\end{quote}

\pnum
Change in 6.9.2.1 {[}intro.races{]} paragraph 21:

\begin{quote}
... {[}Note: It can be shown that programs that correctly use
mutexes\uline{, atomic blocks,} and \texttt{memory\_order::seq\_cst}
operations to prevent all data races and use no other synchronization
operations behave as if the operations executed by their constituent
threads were simply interleaved, with each value computation of an
object being taken from the last side effect on that object in that
interleaving. This is normally referred to as "sequential consistency".
...
\end{quote}

\pnum
Add a new paragraph after 6.9.2.1 {[}intro.races{]} paragraph 21:

\begin{quote}
{[} Note: The following holds for a data-race-free program: If the start
of an atomic block T is sequenced before an evaluation A, A is sequenced
before the end of T, A strongly happens before some evaluation B, and B
is not sequenced before the end of T, then the end of T strongly happens
before B. If an evaluation C strongly happens before that evaluation A
and C is not sequenced after the start of T, then C strongly happens
before the start of T. These properties in turn imply that in any simple
interleaved (sequentially consistent) execution, the operations of each
atomic block appear to be contiguous in the interleaving. -\/- end note
{]}
\end{quote}

\rSec3[intro.progress]{Forward progress} 

\pnum
Change in 6.9.2.2 {[}intro.progress{]} paragraph 1:

\begin{quote}
\sout{The implementation may assume that any thread will eventually do}
\uline{An \emph{inter-thread side effect} is} one of the following:

\begin{itemize}
\tightlist
\item
  \sout{terminate,}
\item
  a call to a library I/O function,
\item
  an access through a volatile glvalue, or
\item
  a synchronization operation or an atomic operation
  \uline{({[}atomics{]})}.
\end{itemize}

\uline{The implementation may assume that any thread will eventually
terminate or evaluate an inter-thread side effect.} {[}Note: This is
intended to allow compiler transformations such as removal of empty
loops, even when termination cannot be proven. --- end note{]}
\end{quote}

\rSec0[stmt.stmt]{Statements}
\rSec1[stmt.pre]{Preamble}  
\pnum
Add a production to the grammar in 8.1 {[}expr.pre{]}:

\begin{quote}
\begin{verbatim}
statement:
      labeled-statement
      attribute-specifier-seqopt expression-statement
      attribute-specifier-seqopt compound-statement
      attribute-specifier-seqopt selection-statement
      attribute-specifier-seqopt iteration-statement
      attribute-specifier-seqopt jump-statement
      declaration-statement
      attribute-specifier-seqopt try-block
      \uline{atomic-statement}
\end{verbatim}
\end{quote}

\rSec1[stmt.dcl]{Declaration statement}  

\pnum
Add a new subclause before 8.8 {[}stmt.dcl{]}:

\begin{quote}
\textbf{8.8 Atomic statement {[}stmt.tx{]}}

\begin{verbatim}
atomic-statement:
     atomic do compound-statement
\end{verbatim}

An \emph{atomic-statement} is also called an \emph{atomic block}.

The start of the atomic block is immediately before the opening
\texttt{\{} of the \emph{compound-statement}. The end of the atomic
block is immediately after the closing \texttt{\}} of the
\emph{compound-statement}. {[} Note: Thus, variables with automatic
storage duration declared in the \emph{compound-statement} are destroyed
prior to reaching the end of the atomic block; see 8.7 {[}stmt.jump{]}.
-\/- end note {]}

A goto or switch statement shall not be used to transfer control into an
atomic block.

If the execution of an atomic block evaluates an inter-thread side
effect (6.9.2.2 {[}intro.progress{]}) or if an atomic block is exited
via an exception, the behavior is undefined.

\emph{Recommended practice:} In case an atomic block is exited via an
exception, the program should be terminated without invoking a terminate
handler (17.9.5 {[}exception.terminate{]}) or destroying any objects
with static or thread storage duration (6.9.3.4 {[}basic.start.term{]}).

If the execution of an atomic block evaluates any of the following
outside of a manifestly constant-evaluated context (7.7
{[}expr.const{]}), the behavior is implementation-defined:

\begin{itemize}
\tightlist
\item
  an \emph{asm-declaration} (9.10 {[}dcl.asm{]});
\item
  an invocation of a function \textbf{other than one of the standard
  library functions specified in (16.4.6.17 {[}atomic.use{]}), unless
  the function is inline} with a reachable definition;
\item
  a virtual function call (7.6.1.3 {[}expr.call{]});
\item
  a function call, \textbf{unless overload resolution selects}

  \begin{itemize}
  \tightlist
  \item
    \textbf{a named function (12.2.2.2.2 {[}over.call.func{]}) or}
  \item
    \textbf{a function call operator (12.2.2.2.3
    {[}over.call.object{]}), but not a surrogate call function;}
  \end{itemize}
\item
  a \texttt{co\_await} expression (7.6.2.3 {[}expr.await{]}), a
  \emph{yield-expression} (7.6.17 {[}expr.yield{]}), or a
  \texttt{co\_return} statement (8.7.4 {[}stmt.return.coroutine{]});
\item
  dynamic initialization of a block-scope variable with static storage
  duration; or
\item
  dynamic initialization of a variable with thread storage duration.
\end{itemize}

{[} Note: The implementation \textbf{can} define that the behavior is
undefined in some or all of the cases above. {]}

{[} Example:

\begin{verbatim}
unsigned int f()
{
  static unsigned int i = 0;
  atomic do {
    ++i;
    return i;
  }
}
\end{verbatim}

Each invocation of f (even when called from several threads
simultaneously) retrieves a unique value (ignoring
\textbf{wrap-around}). -\/- end example {]}

{[} Note: Atomic blocks are likely to perform best where they execute
quickly and touch little data. -\/- end note {]}
\end{quote}

\rSec1[library]{Library introduction}
\rSec2[description]{Method of description} 
%\rsec2[description]{Method of description} 

\pnum

Add 16.4.6.17 {[}atomic.use{]}:

\begin{quote}
\hypertarget{functions-usable-in-an-atomic-block-atomic.use}{%
\subsection{16.4.6.17 Functions usable in an atomic block
{[}atomic.use{]}}\label{functions-usable-in-an-atomic-block-atomic.use}}

All library functions may be used in an atomic block (8.8
{[}stmt.tx{]}), except

\begin{itemize}
\tightlist
\item
  \textbf{error category objects ({[}syserr.errcat.objects{]})}
\item
  time zone database ({[}time.zone.db{]})
\item
  clocks ({[}time.clock{]})
\item
  \texttt{signal} ({[}support.signal{]}) \textbf{and \texttt{raise}
  ({[}csignal.syn{]})}
\item
  \texttt{set\_new\_handler}, \texttt{set\_terminate},
  \texttt{get\_new\_handler}, \texttt{get\_terminate}
  ({[}handler.functions{]}, {[}alloc.errors{]}, {[}exception.syn{]})
\item
  \textbf{\texttt{system} ({[}cstdlib.syn{]})}
\item
  \textbf{startup and termination {[}support.start.term{]} except
  \texttt{abort}}
\item
  \texttt{shared\_ptr} ({[}util.smartptr.shared{]}) and
  \texttt{weak\_ptr} ({[}util.smartptr.weak{]})
\item
  \texttt{synchronized\_pool\_resource} ({[}mem.res.pool{]})
\item
  \textbf{program-wide \texttt{memory\_resource} objects
  ({[}mem.res.global{]})}
\item
  \texttt{setjmp} / \texttt{longjmp} ({[}csetjmp.syn{]})
\item
  \textbf{parallel algorithms ({[}algorithms.parallel{]})}
\item
  \textbf{\texttt{random\_device} ({[}rand.device{]})}
\item
  \texttt{locale} construction ({[}locale.cons{]})
\item
  input/output ({[}input.output{]})
\item
  atomic operations ({[}atomics{]})
\item
  thread support ({[}thread{]})
\end{itemize}
\end{quote}
